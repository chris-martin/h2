\nonstopmode
\title{CS 6422 Project Proposal}
\author{Christopher Martin\\{\tt chris.martin@gatech.edu}}

\documentclass{article}

\begin{document}

\maketitle

\begin{abstract}
Relational databases tend to have weak security features, which can be
problematic for information systems that require strong security guarantees.
This project will extend an open source database system to enable enforcement
of multi-level compartmental mandatory access control policies.

\end{abstract}

\section{Security model}

This project assumes a simplified and generalized version of the rules
used by the U.S. Department of Defense to protect classified data.

\begin{description}

  \item[Sensitivity level] A fully ordered set of labels. Herein we call
    these labels $0$, $1$, $2$, and $3$, where the infimum level
    $0$ denotes the least sensitive data, and the supremum level $3$
    is applied to the most highly-guarded secrets.

  \item[Compartment] A set of labels indicating topics to which data
    pertains. This set is partially ordered because topics may be nested
    hierarchically. For example, if the compartment {\it Food} contains
    the compartments {\it Apples} and {\it Bananas}, then
    ${\it Food} > {\it Bananas}$, but {\it Apples} and {\it Bananas}
    are incomparable.

  \item[Marking] A marking consists of one sensitivity level and any
    number of compartments. We denote this by separating the components
    with slashes, such as {\it 2/Apples/Bananas}. Markings form a lattice
    where the infinum is {\it 0} and the supremum is a marking with the
    highest sensitivity and every compartment.

  \item[Credential] A credential consists of a sensitivity level and
    a compartment. Each subject has a set of credentials.

\end{description}

A subject has read access to data marked by
{\it $\ell$/$C_1$/$C_2$/\ldots/$C_n$} only if for all $i \in [1, n]$
the subject has a credential ($\ell'$, $C'$) such that
$\ell' \ge \ell$ and $C' \ge C_i$.

\section{Implementation}

The deliverable will be a modified version of
H2\footnote{H2 database: http://h2database.com/},
an open source pure-Java DBMS.

H2's internal schema consists of {\tt MetaTable}s, one of which is
for {\tt User}s. This schema will need to be extended to store user
credentials. A user with the {\it modify-credentials} {\tt Role}
will be able to invoke a built-in procedure to modify users'
credentials.

One candidate idea to achieve information hiding is to shadow every
{\tt Table} with a {\tt TableView} that selects only rows that the
user is allowed to access. When the connection is first established,
we can use the user's credentials and the compartment hierarchy to
fill a metatable of credentials that the user does {\it not} have.
With that set precomputed, it should be possible to generate each
table's view by selecting all rows that do not require those credentials.

Oracle has a similar extension to Oracle Database called
Oracle Label Security (OLS)\footnote{
Oracle Label Security:
\url{http://www.oracle.com/us/products/database/options/label-security/overview/index.html}
} which may serve as inspiration.

\begin{description}

  \item[Phase 1] Clone the H2 SVN repository in git, and identify a stable
    version to use as a baseline for modifications.
    Establish a relational schema, test data, and queries to use for performance
    testing.

  \item[Phase 2] Build proof of concept for shadowing table views,
    using a fixed set of user credentials. Add procedures for setting row markings.
    Write tests to demonstrate correctness of security policy enforcement.

  \item[Phase 3] Add user credential management procedures, and initialize
    credential tables when a database connection is established.
    Run performance tests to measure the slowdown introduced by the security checks.

\end{description}

\end{document}
