\nonstopmode
\title{CS 6422 Project Proposal}
\author{Christopher Martin\\{\tt chris.martin@gatech.edu}}

\documentclass{article}

\begin{document}

\maketitle

\begin{abstract}
Relational databases tend to have weak security features, which can be problematic for information systems that require strong security guarantees. This project will extend an open source database system to enable enforcement of multi-level compartmental mandatory access control policies.

\end{abstract}

\section{Security model}

This project assumes a simplified and generalized version of the rules used by the U.S. Department of Defense to protect classified data.

\begin{description}

  \item[Sensitivity level] A fully ordered set of labels. Herein we call these labels $0$, $1$, $2$, and $3$, where the infimum level $0$ denotes the least sensitive data, and the supremum level $3$ is applied to the most highly-guarded secrets.

  \item[Compartment] A set of labels indicating topics to which data pertains. This set is partially ordered because topics may be nested hierarchically. For example, if the compartment {\it Food} contains the compartments {\it Apples} and {\it Bananas}, then ${\it Food} > {\it Bananas}$, but {\it Apples} and {\it Bananas} are incomparable.

  \item[Marking] A marking consists of one sensitivity level and any number of compartments. We denote this by separating the components with slashes, such as {\it 2/Apples/Bananas}. Markings form a lattice where the infinum is {\it 0} and the supremum is a marking with the highest sensitivity and every compartment.

  \item[Credential] A credential consists of a sensitivity level and a compartment. Each subject has a set of credentials.

\end{description}

A subject has read access to data marked by {\it $\ell$/$C_1$/$C_2$/\ldots/$C_n$} only if for all $i \in [1, n]$ the subject has a credential ($\ell'$, $C'$) such that $\ell' \ge \ell$ and $C' \ge C_i$.

\section{Implementation}

The deliverable will be a modified version of H2\cite{h2}, an open source pure-Java DBMS.

H2's internal schema consists of {\tt MetaTable}s, one of which is for {\tt User}s. This schema will need to be extended to store user credentials. A user with the {\it modify-credentials} {\tt Role} will be able to invoke a built-in procedure to modify users' credentials.

One candidate idea to achieve information hiding is to shadow every {\tt Table} with a {\tt TableView} that selects only rows that the user is allowed to access. When the connection is first established, we can use the user's credentials and the compartment hierarchy to fill a metatable of credentials that the user does {\it not} have. With that set precomputed, it should be possible to generate each table's view by selecting all rows that do not require those credentials.

Oracle has a similar extension to Oracle Database called Oracle Label Security (OLS)\cite{ols} which may serve as inspiration.

\pagebreak

\section{Task breakdown}

\paragraph{Setup}

\begin{itemize}
\item Clone the H2 SVN repository in git.
\item Identify a stable version to use as a baseline for modifications.
\end{itemize}

\paragraph{Proof of concept}

\begin{itemize}
\item Build proof of concept for shadowing table views, using a fixed set of user credentials.
\item Write tests to demonstrate correctness of security policy enforcement.
\item Run performance tests to measure the slowdown introduced by the security checks.
\end{itemize}

\paragraph{Evaluation}

\begin{itemize}
\item Demonstrate correctness of security properties
\item Measure performance
\end{itemize}

\paragraph{Extended features}

If time permits, complete the full feature set to make this project truly usable.

\begin{itemize}
\item Add procedures for setting row markings.
\item Add user credential management procedures.
\item When a database connection is established, initialize credential tables for the user.
\end{itemize}

\pagebreak

\section{Evaluation plan}

The test setup models a system for storing multi-page documents.

\paragraph{Person}
\begin{verbatim}
  user_id      int       pk
  user_name    varchar
\end{verbatim}

\paragraph{Document}
\begin{verbatim}
  doc_id       int       pk
  title        varchar
  released     date
  author       int       fk: Person.user_id
\end{verbatim}

\paragraph{Page}
\begin{verbatim}
  doc_id       int       fk: Document.doc_id
  page_number  int
  page_text    clob
\end{verbatim}

The {\tt Document} and {\tt Page} tables are both separately protected with markings, so a user who can see a document may not be able to see all of its pages.\footnote{It likely doesn't make sense for a user to have access to a {\tt Page} without also having access to its {\tt Document}, but the database system will not enforce such a rule.}

\paragraph{Correctness}

The foremost question will be whether the security properties hold, which is a simple matter of applying various document and page markings and verifying that {\tt select} queries executed by different database users return appropriate subsets of the data.

\paragraph{Performance}

Performance evaluation will test {\tt insert} and {\tt select} queries to answer two questions:

\begin{itemize}
\item What is the base cost incurred by enabling security in the most trivial case where all tuples are marked with the minimum sensitivity and no compartments?
\item How does performance scale as a function of the number of tuples and the number of security compartments, in comparison with the same system without the security feature?
\end{itemize}

\end{description}

\bibliographystyle{plainnat}
\begin{thebibliography}{99}
\bibitem{h2}{H2 database. \url{http://h2database.com/}}
\bibitem{ols}{Oracle Label Security. \url{http://www.oracle.com/us/products/database/options/label-security/overview/index.html}}
\end{thebibliography}

\end{document}
